%% LyX 2.0.5.1 created this file.  For more info, see http://www.lyx.org/.
%% Do not edit unless you really know what you are doing.
\documentclass[ngerman]{article}
\usepackage[T1]{fontenc}
\usepackage[latin9]{inputenc}
\usepackage{array}

\makeatletter

%%%%%%%%%%%%%%%%%%%%%%%%%%%%%% LyX specific LaTeX commands.
%% Because html converters don't know tabularnewline
\providecommand{\tabularnewline}{\\}

%%%%%%%%%%%%%%%%%%%%%%%%%%%%%% User specified LaTeX commands.
\date{}

\makeatother

\usepackage{babel}
\begin{document}

\title{Instruktionen (baseline market)}

\maketitle

\section{Marktsimulation}

In den n�chsten ca. 90 min werden Sie an einer Marktsimulation teilnehmen.
Die Regeln sind f�r alle Teilnehmer gleich. Ihre Entlohnung ist abh�ngig
von den Entscheidungen, die Sie w�hrend der Simulation treffen.

Die Marktsimulation besteht aus 15 Runden, wobei jede Runde genau
3 Minuten (also 180 Sekunden) dauert. W�hrend der Simulation wird
in ECU (experimental currency units) gehandelt. 100 ECU entsprechen
1 Euro. 

Jeder Spieler startet die erste Runde mit einem Geldkapital von 720
ECU und 4 Aktien. W�hrend jeder Runde k�nnen Sie Aktien zum Verkauf
anbieten oder kaufen. Wenn Sie Aktien zu einem Preis kaufen, dann
wird dieser Preis von Ihrem Geldkapital abgezogen. Wenn Sie eine Aktie
verkaufen, wird der Preis zu Ihrem Geldkapital hinzugef�gt. Sie k�nnen
zu keinem Zeitpunkt mehr Geld f�r eine Aktie bezahlen als Sie Geldkapital
haben. Es ist auch nicht m�glich eine Aktie zum Verkauf anzubieten,
wenn Sie keine Aktie besitzen.

Am Ende jeder Runde erhalten sie pro Aktie zus�tzliches Geldkapital.
Wie viel Geldkapital Sie f�r Ihre Aktien am Ende der Runde erhalten,
wird vom Computer zuf�llig entschieden. M�glich sind 0, 8, 28 und
60 ECU pro Aktie. Jede M�glichkeit ist gleich wahrscheinlich und gilt
f�r alle Ihre Aktien in dieser Runde. Durchschnittlich erhalten Sie
24 ECU pro Runde pro Aktie. Wenn Sie z.B. eine Aktie 7 Runden vor
Schluss kaufen und nicht mehr verkaufen, bekommen Sie durchschnittlich
7{*}24 = 168 ECU f�r die Aktie. Zur besseren �bersicht finden Sie
auf der letzten Seite der Instruktionen eine Tabelle die den durchschnittlichen
Wert einer Aktie in jeder Runde zeigt. Ihre Entlohnung entspricht
am Ende der Simulation ihrem Geldkapital, umgerechnet in Euro.


\section{Computerprogramm}

Die Marktsimulation findet mit Hilfe eines Computerprogramms statt.
Wir erkl�ren jetzt die einzelnen Funktionen des Programms. Sie werden
gebeten Aktionen zur Probe auszuf�hren. Diese Aktionen geh�ren noch
nicht zur Marktsimulation und haben keinen Einfluss auf ihre Entlohnung.

Im oberen Teil des Bildschirmes sehen Sie das Geldkapital (720) das
Ihnen zur Verf�gung steht, rechts daneben wie viele Aktien (4) Sie
besitzen.

In der Spalte ganz links befindet sich ein Eingabefeld �ber dem ,,Preis
zu dem ich verkaufe'' steht. In dem Eingabefeld k�nnen Sie einen
Preis eingeben zu dem Sie eine Aktie verkaufen wollen. Dieses Angebot
wird erst an die anderen Spieler geschickt wenn Sie auf das rote Feld
unten ,,Aktie anbieten'' klicken. \emph{Geben Sie nun eine Zahl
von maximal 720 ein und klicken Sie auf ,,Aktie anbieten''}.

Sie sehen nun ebenso viele\emph{ }Zahlen in der Spalte rechts daneben
(,,Angebotene Aktien'') wie es Teilnehmer an der Marktsimulation
gibt. Dies sind alle Aktien, die Spieler zu einem entsprechenden Preis
anbieten. Die Preise sind in aufsteigender Reihenfolge von Unten nach
Oben sortiert. Durch klicken auf einen Preis wird dieser farblich
hervorgehoben und ausgew�hlt. Durch klicken auf das Feld ,,Kaufen''
kaufen Sie die farblich hervorgehobene Aktie zu dem entsprechenden
Preis. Ihr eigenes Angebot steht in blauer Schrift. Dieses Aktie k�nnen
Sie nicht kaufen. \emph{Bitte kaufen Sie nun eine Aktie}.

Sie sollten nun wieder 4 Aktien haben, da Sie eine verkauft und eine
gekauft haben.

In der ganz rechten Spalte k�nnen Sie ein Angebot machen, zu dem andere
Spieler Aktien an Sie verkaufen k�nnen. Hierzu geben Sie einen Preis
ein, der maximal Ihrem Geldkapital entspricht. Dieser Preis wird erst
an die anderen Spieler geschickt wenn Sie auf das rote Feld ,,Preis
bieten'' klicken. \emph{Geben Sie nun einen Betrag ein und klicken
Sie auf das rote Feld ,,Preis bieten''.}

Sie sehen nun ebenso viele\emph{ }Zahlen in der Spalte links (,,Gebotene
Preise'') wie es Teilnehmer in der Marktsimulation gibt. Dies sind
Preise die Spieler f�r eine Aktie bieten. Die Preise sind in aufsteigender
Reihenfolge von Unten nach Oben sortiert. Durch klicken auf einen
Preis wird dieser farblich hervorgehoben und ausgew�hlt. Durch klicken
auf das Feld ,,Verkaufen'' verkaufen Sie eine Ihrer Aktien f�r den
ausgew�hlten Preis an den Spieler der das Angebot gemacht hat. \emph{Bitte
w�hlen Sie nun einen Preis und verkaufen Sie eine Aktie.}

In der mittleren Spalte werden alle bisherigen Aktien Verk�ufe mit
ihren dazugeh�rigen Preisen gelistet. Der erste Handel steht oben,
der Letzte unten.

Sie werden nun gemeinsam eine �bungsrunde haben. Diese �bungsrunde
ist in Dauer und Art genauso wie die sp�teren Runden in der echten
Marktsimulation. Diese �bungsrunde hat KEINE Auswirkungen auf die
Marktsimulation oder Ihre Entlohnung und wird nicht gespeichert oder
bewertet.


\section{Nichtteilnehmen}

Zu Beginn jeder Periode werden Sie gefragt, ob Sie an der n�chsten
Runde teilnehmen wollen. Antworten Sie mit Nein, k�nnen Sie weder
kaufen oder verkaufen, noch sehen wie Aktien gehandelt werden. Ihre
Entscheidung beeinflusst nicht, wie viel Sie am Ende der Runde pro
Aktie ausgezahlt bekommen. Entscheiden Sie sich f�r die Teilnahme
an einer Runde und haben am Ende der Runde 1 Aktie, dann ist bringt
diese Aktie durchschnittlich genauso viel ein, wie wenn Sie 1 Aktie
zu Beginn der Runde haben und entscheiden nicht an der Runde teilzunehmen.
Weder eine Teilnahme noch ein Nichtteilnahme ist besser f�r Ihre Entlohnung.
Die Entscheidung steht Ihnen v�llig frei.

\newpage{}

\begin{table}
\caption{Zu erwartende Einnahmen pro Aktie in Ihrem Besitz}
\begin{tabular}{|>{\centering}p{2cm}||>{\centering}p{2.8cm}|>{\centering}p{0.5cm}|>{\centering}p{2.8cm}|>{\centering}p{0.5cm}|>{\centering}p{2.8cm}|}
\hline 
Aktuelle Runde & Restliche Auszahlungen bis Ende der Simulation & {*} & Durchschnittliche Auszahlung pro Aktie pro Runde (In ECU) & = & Durchschnittliche Auszahlung pro Aktie bis Ende der Simulation (In
ECU)\tabularnewline
\hline 
\hline 
1 & 15 & {*} & 24 & = & 360\tabularnewline
\hline 
2 & 14 & {*} & 24 & = & 336\tabularnewline
\hline 
3 & 13 & {*} & 24 & = & 312\tabularnewline
\hline 
4 & 12 & {*} & 24 & = & 288\tabularnewline
\hline 
5 & 11 & {*} & 24 & = & 264\tabularnewline
\hline 
6 & 10 & {*} & 24 & = & 240\tabularnewline
\hline 
7 & 9 & {*} & 24 & = & 216\tabularnewline
\hline 
8 & 8 & {*} & 24 & = & 192\tabularnewline
\hline 
9 & 7 & {*} & 24 & = & 168\tabularnewline
\hline 
10 & 6 & {*} & 24 & = & 144\tabularnewline
\hline 
11 & 5 & {*} & 24 & = & 120\tabularnewline
\hline 
12 & 4 & {*} & 24 & = & 96\tabularnewline
\hline 
13 & 3 & {*} & 24 & = & 72\tabularnewline
\hline 
14 & 2 & {*} & 24 & = & 48\tabularnewline
\hline 
15 & 1 & {*} & 24 & = & 24\tabularnewline
\hline 
\end{tabular}

\end{table}

\end{document}
