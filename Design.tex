%% LyX 2.0.5.1 created this file.  For more info, see http://www.lyx.org/.
%% Do not edit unless you really know what you are doing.
\documentclass[english]{article}
\usepackage[T1]{fontenc}
\usepackage[latin9]{inputenc}
\usepackage{babel}
\begin{document}

\title{The effect of algorithmic trading in asset markets on human willingness
to trade (Handout)}


\author{Mike Farjam}

\maketitle

\section{Research Question}


\subsection{Main}

Does the presence of \emph{algorithmic traders} (AT) in an asset market
lower the willingness of subjects to trade in the market?


\subsection{Minor}

Do people that deliberately trade in asset markets with algorithmic
trading differ from people that deliberately trade in markets with
only human participant?
\begin{enumerate}
\item Differ with respect to their \emph{risk preference}
\item Differ with respect to their \emph{overconfidence}
\end{enumerate}

\section{Expectations}

We expect that more subjects will refuse to trade/trade less in markets
with AT, than in markets without AT. Those that still chose to trade
in markets with AT are on average less risk averse and more overconfident
than average subjects that chose to trade in markets without AT.


\section{Markets}


\subsection{Baseline condition}

Based on Smith, Nonesuch, and Williams (1988). Subjects trade assets
that pay a random dividend \{0, 8, 28, 60\} per round and can trade
those assets in a continuous double auction. 15 rounds are played,
so the fundamental value per asset at the beginning of the simulation
is 15 {*} mean(0, 8, 28, 60) = 360 units and decreasing by mean(0,
8, 28, 60) with every round. Subjects start with 720 ECU cash which
they can use to buy assets sold buy other participants. Short-selling
is not allowed. Each period takes 3 minutes.


\subsection{Experimental condition}

Starting with a random draw. With probability 0.5 subjects end up
(during the entire experiment) in a market that is identical to the
baseline market. With the same probability subjects end up in a market
in which there is AT. Subjects are informed about the probability,
but not about the result of the random draw. Hence subjects are uncertain
if they will trade with AT in this condition. After the simulation
subjects are asked whether they believe that AT were in the market.


\section{Algorithmic Traders}

After the baseline condition subjects will be asked to imagine how
algorithmic traders (AT) would trade in such a market and to write
down at most 3 adjectives with which they would describe those AT.
These adjectives will be used in the experimental condition to describe
to subjects how the AT are programmed%
\footnote{Of course we program the AT in such a way that it can be described
by the adjectives%
}. This way of introducing AT is chosen because by describing AT we
control for believes of subjects about AT. At the same time we do
not change the average subjects' believes about AT (assuming equal
believes about AT in the conditions because of randomization).

In the baseline condition we will measure the time that it took for
a contract to get from the status ``offered'' to ``sold''. AT
in the experimental condition will not react immediately to offers
made by other market participants. Instead they wait with considering
offers made on the market with a delay that is equal to the average
time that conclusion of a contract took in the baseline condition.
We chose this action to make it not too obvious for subjects whether
they are in an market wit AT. Subjects are informed on the latency
of AT.


\section{Measurements}


\subsection{Risk preference}

Before the market simulation we measure subjects risk preference as
proposed by Holt and Laury (2002).


\subsection{Overconfidence}

There will be two overconfidence measurements.
\begin{enumerate}
\item Subjects are asked after the Holt and Laury task to guess the average
switching point (from lottery A to lottery B) of other participants
in the group. Subjects are incentivised for the precision of their
guess. After response they are asked to indicate the confidence that
their guess is correct\emph{. }
\item Subjects answer 7 riddles (30 seconds time for answer per riddle)
taken from Bruguier et al. (2010) and are incentivised for correct
answers. After the riddles they are asked how many of the questions
they think they answered correctly and asked to indicate the confidence
that their guess is correct.
\end{enumerate}

\subsection{Trading}

Per subject we measure the number of buying and selling actions that
were made in every period.


\subsection{Outside option}

At the beginning of each round subjects can push a button that will
lock the screen for the next round, disenabling trade for that period.
Choosing the outside option will automatically set the number of buying
and selling actions to zero.%
\footnote{In the instructions we emphasize that both options, trading and the
outside option, may be chosen equivalently, are adequate actions and
that the subject should chose what he/she prefers.%
}


\section{Variables and operationalizations}


\subsection{Dependend variables}
\begin{enumerate}
\item Number of subjects locking the screen per period.
\item Number of trades per subject.
\end{enumerate}

\subsection{Independend variables}
\begin{enumerate}
\item Potential presence of algorithmic trading
\item Risk preference
\item Overconfidence
\end{enumerate}

\section{Payment}
\begin{enumerate}
\item Assuming risk neutrality and rationality subjects earn in the \emph{Holt
and Laury lottery} on average: 5.175 Euro
\item If subjects\emph{ guess the average switching point} for the Holt
and Laury task correctly they get 1 Euro extra.
\item Per \emph{riddle} subjects can earn 50 cent for the correct answer.
So ideally subjects earn 3.5 Euro for all riddles.
\item If subjects \emph{guess the number of correct answers} for the riddle
task correctly they get 1 Euro extra.
\item On average every subjects can earn during the \emph{market simulation}:\linebreak{}
$initialCash+initialShares*rounds*averageDividend=2160$ ECU\end{enumerate}

\end{document}
