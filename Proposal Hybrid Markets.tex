%% LyX 2.0.5.1 created this file.  For more info, see http://www.lyx.org/.
%% Do not edit unless you really know what you are doing.
\documentclass[english,doc]{apa}
\usepackage[T1]{fontenc}
\usepackage[latin9]{inputenc}
\usepackage{graphicx}

\makeatletter
%%%%%%%%%%%%%%%%%%%%%%%%%%%%%% Textclass specific LaTeX commands.
\helvetica
\author{Author} % hack around some bugs in apa.cls
\affiliation{Affiliation} % hack around some bugs in apa.cls

%%%%%%%%%%%%%%%%%%%%%%%%%%%%%% User specified LaTeX commands.
%% uncomment the following line to get this information
% below the tile
%\note{Draft of \today}

%% Some information on the journal. This will be
%% printed on the headline of the first page
%% Just uncomment wir a '%' if you don't want this
% Journal name
\journal{The Test Journal}
% volume, number, pages
\volume{Vol. 0 (2007), pp.~1--22}
% copyright notice
\ccoppy{\textcopyright\ The Author}
% Serial number or other reference
\copnum{ISSN XXX-XXXX-XXXX}

%% The usual ...
\acknowledgements{Author notes, acknowledgements, contact information go here \ldots}

\note{This is a note}

\makeatother

\usepackage{babel}
\begin{document}

\title{Changes in human behavior under the presence of algorithmic traders
in experimental asset markets}


\rightheader{Presence of algorithmic traders in asset markets}


\author{Mike Daniel Farjam}


\affiliation{Friedrich Schiller University Jena, International Max Planck Research
School on Adapting Behavior in a Fundamentally Uncertain World, Germany}


\abstract{bladibla}

\maketitle
In 1988, Smith, Suchanek, and Williams (SSW) published work in experimental
asset markets that became canonical in experimental economics. In
these markets assets that pay a random dividend are traded in an anonymized
continuous double action. Subjects start with a stack of assets and
some cash which they can use to buy assets that are available on the
market. The average dividend that an asset pays and the time available
for trading is known to subjects. Thus subjects (theoretically) know
the fundamental value of an asset in these markets. SSW find that
asset prices in the experimental markets follow a \textquotedblleft{}bubble
and crash\textquotedblright{} pattern that is close to speculative
bubbles in real world asset markets. Many modifications of the SSW
design have been used to understand the reasons for market bubbles
and design markets that are less probable to produce such bubbles
(for a full survey see Palan (2013)). 

The biggest change in real world asset markets since the 1980s is
the still increasing use of automated \emph{algorithmic trading} with
the help of computers. The use of computers in asset markets comes
in many flavors. Starting with scheduling of sales for large amounts
of assets without influencing the asset price on the market, but also
including fully automated strategic algorithms that learn and replace
human traders in the decision on what to sell and buy on the markets
(Kirilenko and Lo, 2013). It is estimated that AT is responsible for
up to 70\% of the total trading volume in major European and US equity
exchanges (De Luca et al., 2011). 

Despite this major change in real-world asset markets, it seems that
experimental economists prefer to study (human-only) markets of the
1980s to propose policies for the (hybrid human-computer) markets
of the 21st century. A very small amount of studies have tried to
study hybrid markets. The major part of these studies focuses on the
computer side of hybrid markets. Das et al. (2001) and De Luca and
Cliff (2011) did \textquotedblleft{}horse race\textquotedblright{}
studies in which they compare the performance of different kinds of
algorithms with the performance of humans. Other experiments try to
identify properties in which hybrid markets differ from human-only
markets. Walsh et al. (2013) find that liquidity improves in hybrid
markets. Gsell (2008) shows with a simulation approach that AT leads
to a higher volatility of the market. In an empirical analysis of
the NYSE since 2003 Hendershott, Jones, and Menkveld (2011) find that
liquidity increased on the market as the use of AT increased. Furthermore,
they find an increase in speed of price discovery. 

All of these studies ignore the possibility that differences between
human-only and hybrid markets may not just be due to different properties
of algorithmic compared to human traders, but may also be a result
of the different expectations that humans have in hybrid markets with
respect to participants in the market they may interact with (namely
computers). 

In this paper, we study empirically the changes in human market behavior
when humans believe that there is AT in the market. To the best of
our knowledge there is only one study with a comparable research question.
Grossklags and Schmidt (2006) compare experimental asset markets in
which humans trade in hybrid markets without knowing that AT is part
of the market, with hybrid markets in which the presence of AT is
common knowledge. Market prices convert closer to the equilibrium
when the presence of AT is known and markets in which humans know
about it are also more efficient. They find no significant difference
in trading volume between the two markets (although the trend is that
there is less trading when subjects are informed about the presence
of AT). 

Unfortunately Grossklags and Schmidts\textquoteright{} paper has,
from an economic point of view, some weaknesses that we want to overcome.
They do not give all payoff relevant information to subjects in the
no-information condition (namely the presence of AT) and they do not
control for the believes that subjects may have about AT. 

While Grossklags and Schmidt focus on the aggregate level of the market
in their analysis, we want to learn more about the subjects on an
individual level. What kind of personalities chose to trade in hybrid
markets? More precisely, what is their\emph{ risk preference} and
level of \emph{overconfidence}? Is there a critical chance to interact
with algorithmic traders in the market at which subjects prefer not-trading
above trading in the hybrid market? In the following section we will
review literature in standard (human-only) market experiments that
is relevant for similar questions in hybrid markets and we derive
hypotheses for our experiment from that literature. In the method
section we describe the concrete design we chose to test the hypotheses. 


\section{Relevant Literature}

Cheung et al. (2012) report an experiment in which they trained all
subjects in an asset market similar to the SSW market. Either subjects
were informed after the training that all participants of the market
were trained in the same way (experimental condition), or this information
was not given (baseline condition). They find that markets in which
subjects were informed about the training were less likely to produce
bubbles and mispricing. Their conclusion is that common knowledge
of rationality decreases bubbles. It is likely that participants were
trading less in the informed condition since trading volume and mispricing
are known to be correlated (REFERENCE TO COME), but unfortunately
Cheung et al. do not report explicitly on this. We assume that algorithmic
traders are seen as more rational and less driven by emotion than
humans and that mispricing and trading will therefore be less when
subjects know that they are trading in a hybrid market (Hypothesis
1). 

In our experiments subjects will have an outside option to leave the
market for one period and fix there asset and cash stack for that
time. Lei et al. (2001) find that such an outside option decreases
the size of bubbles (but not eliminates them completely). We interpret
this outside option as a strong form of not-trading and assume that
subjects in hybrid markets will prefer the outside option more often
above trading than subjects in human-only markets (Hypothesis 2). 

We want to get extra information on the subjects that chose to trade
(more) in hybrid markets. For this purpose we will measure two traits
of the subjects: \emph{Risk preference} and \emph{overconfidence}.
The general finding is that a low risk preference in markets leads
to smaller bubbles and less trade. Robin et al. (2012) and Fellner
and Maciejovsky (2007) tested this experimentally in SSW markets and
used the Holt and Laury (2002) procedure to measure risk aversion.
Keller and Siergist (2006) did a mail survey and found that financial
risk tolerance is a predictor for the willingness to engage in asset
markets. We expect to reproduce these findings in our markets (Hypothesis
3a) and furthermore, we expect that the percentage of trade done by
risk averse subjects is even smaller in hybrid markets than in human-only
markets (Hypothesis 3b).

Odean (1999) assumes that overconfidence of traders the reason that
there is more trade than one would expect if all traders were rational.
Michailowa (2010; 2011), Fellner and Kr�gel (2012a), and Oechssler
et al. (2011) find that the size of bubbles and trading activity in
SSW markets are strongly correlated with overconfidence. However,
other authors find no or only very weak correlations with overconfidence
(e.g. Glaser and Weber, 2007; Biais et al., 2005). It seems that the
operationalization of the (not precisely defined) term \textquotedblleft{}overconfidence\textquotedblright{}
matters. Moore and Healy (2008) and Hilton et al. (2011) describe
different operationalizations of overconfidence. Fellner and Kr�gel
(2012b) point out that well established measures of overconfidence
from cognitive psychology \textendash{} such as the miscalibration
measure \textendash{} are not measuring overconfidence as it is used
in economic models. Hence, we chose to operationalize overconfidence
specifically in the context of asset market.

Buying assets for a price above the fundamental value is only rational
if one is confident that one can sell those assets for an even higher
price in the future. SSW and Haruvy et al. (2007) find that subjects
form expectations of future asset prices, but are not able to predict
turning points of prices and underestimate the violence of a price
crash. In this line of thought one needs to be better than other traders
with respect to two abilities: the ability to predict decisions of
others (when will they stop buying) and better analytical/mathematical
skills than others (what is the fundamental value of the asset). As
found earlier, we expect that overconfidence of subjects positively
correlates with trading activity (Hypothesis 4a). Furthermore, we
expect that the percentage of trade done by overconfident subjects
is larger in hybrid markets than in human-only markets. (Hypothesis
4b).


\section{Methods}


\subsection{Riskpreference and overconfidence}

To measure risk aversion of subjects we use the multiple price list
method by Holt and Laury (2002). In this tasks subjects choose between
a lottery with high variance of payoffs (Option B) and lottery with
less variance (Option A). Table 1 gives an overview of all 10 choices
(one per row) that subjects have to make and the respective probabilities
of payoffs in these lotteries. At the end of the experiment one of
the 10 choices is randomly used to compute the payoff for this task.
In this task subjects typically switch in one row from Option A to
Option B, different switching points represent different risk preferences.

\begin{figure}
\begin{minipage}[t]{1\columnwidth}%
\begin{center}
Table 1: Set of binary lotteries used for the Holt-Laury task.
\par\end{center}%
\end{minipage}

\includegraphics[clip,scale=0.8]{\string"table 1\string".eps}
\end{figure}


Since there is no clear preference in the overconfidence literature
on one task and the overconfidence construct has many dimensions,
we chose to measure two of these dimensions with one task each. First
we want to measure how confident a subject is that it is able to predict
the risk preference of others. For this purpose we simply ask subjects
after the risk measurement, what they believe was the average switching
point of the other subjects and how confident they are that their
estimation is correct. The second measurement refers to the subject\textquoteright{}s
mathematical skills. Subjects do a math-quiz and have to state their
expected rank among all participants with respect to their performance
in the quiz and how confident they are about their assumed rank. Overconfidence
in this context is defined as \textquotedblleft{}real rank\textquotedblright{}
minus \textquotedblleft{}assumed rank\textquotedblright{}. Both confidence
measurements are incentivized (for precision of the estimations and
performance in the math task). (We take the logics test as it was
used by Bruguier et al. (2010). One may question whether this test
really measures arithmetic skills as they are needed during the experiment.
However, we do not tend to really measure mathematical skill but are
interested in how the subject compares itself w.r.t. mathematical
skills of others in general.)


\subsection{Baseline Market}

After measurements of risk preference and overconfidence subjects
are randomly assigned either to the baseline market- or the random
draw condition (see next section). In the baseline condition subjects
trade assets in a continuous double auction. Markets replicate those
presented by SSW. As in SSW subjects will trade during 15 periods
and dividend per period is uniform over potential outcomes 0, 8, 28,
or 60 cent. The average dividend per round is thus 24 cent and the
fundamental value of an asset is 360 at beginning of the market simulation
and will decrease by 24 after each period. Every subject starts in
period 1 with a stack of 4 assets and 720 cent in cash, which they
can use to buy assets that are offered on the market. Trading on the
market is anonymized. Each period lasts 3 minutes, so that the entire
market simulation takes 45 minutes. 

After the simulation subjects in this condition are finally asked
to write down 3 adjectives that they associate with computerized trading
in real world asset markets. These adjectives will be used in the
Random-Draw condition. We use the Z-Tree platform (Fischbacher, 2007)
for all market simulations, measurements, and questionnaires.


\subsection{Random Draw}

As suggested by the name this condition starts with a random draw
about which subjects are informed. With a 50\% chance subjects will
participate in a market where algorithmic traders are active. We will
refer to this as the random-draw AT (RAT) condition. With a 50\% chance
subjects participate in a market that is identical to the baseline
market. This will be called random-draw human (RH) condition. Subjects
are not informed on the result of the random draw. In all other aspects
RAT and RH are identical to baseline markets. RH and baseline markets
thus only differ with respect to the believe of subjects that there
could be AT in the market. (In the result section we will compare
RAT with baseline markets to study the effect that the believe that
AT is in the market has, keeping all other variables constant.)

In RAT and RH subjects get information on the algorithmic traders
they may interact with. Giving this information comes as a double-edged
sword. On the one hand, it is crucial to keep experimental control
about the believes of subjects. On the other hand, we do not want
to manipulate the believes of subjects in such a way that we induce
an experimental demand that subjects behave the way we expect them
to in our hypotheses. We therefore chose to collect and aggregate
the believes of subjects in the baseline condition and present them
to subjects in RAT and RH as the way that AT is implemented in the
markets they will trade in. Assuming that the believes about AT in
the baseline and RAT and RH are the same, we do not change their believes,
but still control them.


\subsection{Algorithmic Traders }

to come


\subsection{Outside Option}

In all markets subjects will have an outside option that they can
chose every period instead of participating in the asset market. This
option is offered before the start of a period. If chosen the screen
will turn black for the duration of the period. We interpret this
a choice for this option as the strong form of not-trading.


\subsection{Subjects}

Markets with males are known to produce bigger bubbles (Reference).
Furthermore, women are more risk averse (Reference). We therefore
matched groups with respect to gender of market participants. Candidates
that have participated in experimental asset markets were excluded
from the experiment.


\section{Results}

And what came out \ldots{}


\subsection{Descriptives}

This \ldots{}


\subsection{Analysis}

\ldots{} and that.


\section{Discussion}

\bibliographystyle{plainnat}
\nocite{*}
\bibliography{hybridMarkets}

\end{document}
